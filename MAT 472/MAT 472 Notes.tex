\documentclass{article}
\usepackage{../coursenotes}

\title{MAT 472 - Intermediate Real Analysis I}
\author{
{\Large Instructor: Dr. Steven Kaliszewski} \\
		Notes written by Brett Hansen
}
\date{}

\begin{document}

\maketitle
\tableofcontents
\break

\section{Week of August 21st, 2016}
\subsection{The Completeness Property of Real Numbers}
$\mathbb{R}$ and $\mathbb{Q}$ are ordered fields.
\begin{itemize}
\item $\ssor\mathbb{Q}\ssor < \ssor\mathbb{R}\ssor$
\item $\mathbb{Q}$ is countable, $\mathbb{R}$ is uncountable (cardinality)
\item $x^2-2=0$ has solutions in $\mathbb{R}$ but none in $\mathbb{Q}$
\item $\mathbb{R}$ has no gaps (order)
\end{itemize}

\subsubsection{Axiom of Completeness}
Every non-empty subset of $\mathbb{R}$ that has an upper bound has a least upper bound. Along with the greatest lower bound, these bounds can be classified using the following definitions: \\

\begin{defn}[Upper Bound]Let the set $S$ be a non-empty subset of $\mathbb{R}$ and suppose $b\in\mathbb{R}$. $b$ is an upper bound of the set $S$ if $\forall x\in S,\ x \leq b$.
\end{defn}
\begin{defn}[Lower Bound]Let the set $S$ be a non-empty subset of $\mathbb{R}$ and suppose $b\in\mathbb{R}$. $b$ is a lower bound of the set $S$ if $\forall x\in S,\ x \geq b$.
\end{defn}
\begin{defn}[Supremum]Let $B$ be the set of upper bounds of the non-empty subset, $S$, of $\mathbb{R}$ and suppose $b\in B$. If $\forall x\in B,\ b \leq x$, then $b$ is the least upper bound of $S$, or \textit{supremum} of $S$. This can be denoted as $\text{sup}\ S=b$.
\end{defn}
\begin{defn}[Infinum]Let $B$ be the set of lower bounds of the non-empty subset, $S$, of $\mathbb{R}$ and suppose $b\in B$. If $\forall x\in B,\ b \geq x$, then $b$ is the greatest lower bound of $S$, or \textit{infinum} of $S$. This can be denoted as $\text{inf}\ S=b$.
\end{defn}
\example{$A=[0,\ 2]\quad\longrightarrow\quad\text{sup}\ A=2$ \\
		   $A=[0,\ 2)\quad\longrightarrow\quad\text{sup}\ A=2$}

\begin{defn}[Maximum]If the supremum of a set is also a member of the set, then the supremum is also the maximum.
\end{defn}
% \example{Let $A=\{1/n\ssor n\in\mathbb{N}\}$, then $\text{max}\ A=\text{sup}\ A=1$ and $\text{inf}\ A=0$, however there is no minimum.}

% \noindent There are non-empty bounded subsets of $\mathbb{Q}$ with no supremum (or infinum) in $\mathbb{Q}$.
% \example{Let $S=\{r\in\mathbb{Q}\ssor r^2<2\}$. There does not exist any $b\in\mathbb{Q}$ such that $b \leq c\ \text{where}\ c\in\text{UB}\{S\}\ \text{and}\ c\in\mathbb{Q}$.}

% \proof{$S$ is non-empty and bounded above by $2$. Suppose $x\in\mathbb{Q}$ and $x>2$. Then $0\leq 2<x$ so $0\leq 2\cdot 2<2x$ and $0\leq 2x<x^2$. Then $2<4<x^2$ and thus $x\notin S$.}

\section{Week of August 28th, 2016}
\subsection{Cardinality}
\begin{defn}[Finite]
A set, $S$, is finite if $S=\emptyset$ or if there exists a bijection, $f: \{1,2,\ellipsis,n\}\longrightarrow S$ for some $n\in\mathbb{N}$.
\end{defn}
\begin{defn}[Infinite]
A set, $S$, is infinite if it is not finite.
\end{defn}
For sets $A$ and $B$, $A\sim B$ if there exists a bijection, $f:A\longrightarrow B$. $\sim$ is an equivalence relation (symmetric, reflexive, and transitive).
\begin{defn}[Finite]
A set, $S$, is finite if $S=\emptyset$ or if $S\sim\{1,2,\ellipsis,n\}$ for some $n\in\mathbb{N}$.
\end{defn}
\begin{defn}[Countable]
A set, $S$, is countable if $S\sim \mathbb{N}$.
\end{defn}
\begin{defn}[Uncountable]
A set, $S$, is uncountable if it is infinite but not countable.
\end{defn}
\begin{theorem}
$\mathbb{Q}$ is countable, but $\mathbb{R}$ is uncountable.
\end{theorem}
\begin{proof}$\mathbb{R}$ is uncountable. \\
Suppose it is not, and so $\mathbb{R}$ is either countable or finite. It is not finite, but this will be left unproven. So then there must exist a function, $f:\mathbb{N}\longrightarrow\mathbb{R}$ that is one-to-one and onto. Choose a closed and bounded interval, $I$, that doesn't contain $f(1)$. Recursively construct a closed and bounded interval, $I_n,\ n\in\mathbb{N}$ with $I_n\geq I_{n+1}$ and $f(n)\in I_n\ \forall n$. By the nested interval property, $\parunion{n\in\mathbb{N}}{}{I_n}\neq\emptyset$. Now choose $x\in\parunion{n\in\mathbb{N}}{}{I_n}$ and note that $f(x)\neq x\ \forall n$. This is a contradiction and thus $\mathbb{R}$ is uncountable.
\end{proof}

\begin{theorem}
Suppose $A_n$ is countable $\forall n\in\mathbb{N}$. Then,
\begin{enumerate}[label=\roman*)]
\item $\parunion{n=1}{N}{A_n}$ is countable $\forall N\in\mathbb{N}$ (finite union of countable sets is countable), and
\item $\parunion{n\in\mathbb{N}}{}{A_n}$ is also countable (countable union of countable sets is countable).
\end{enumerate}
\end{theorem}

\begin{proof}(?)
\begin{enumerate}[label=\roman*)]
\item First, we know that $A_1\cup A_2$ is countable. Equivalently, $\{\text{evens}\}\cup\{\text{odds}\}\sim\mathbb{N}$ or $f:A_1\longrightarrow\mathbb{N}\longrightarrow \{\text{evens}\}$ and $g:A_2\longrightarrow\mathbb{N}\longrightarrow \{\text{odds}\}$. Thus, if $h:A_1\cup A_2\longrightarrow\mathbb{N}$, then $h(x)=\{f^{\sim}(f(x))\ x\in A_1, g^{\sim}(g(x))\ x\in A_2\}$.
\item $\mathbb{N}\sim\mathbb{N}\times\mathbb{N}$
\end{enumerate}
\end{proof}
\section{Week of September 4th, 2016}
\section{Week of September 11th, 2016}
\section{Week of September 18th, 2016}
\section{Week of September 25th, 2016}
\section{Week of October 2nd, 2016}
\section{Week of October 9th, 2016}
\section{Week of October 16th, 2016}
\section{Week of October 23rd, 2016}
\section{Week of October 30th, 2016}
\section{Week of November 6th, 2016}
\section{Week of November 13th, 2016}
\section{Week of November 20th, 2016}
\section{Week of November 27th, 2016}

\end{document}