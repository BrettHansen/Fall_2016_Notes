\documentclass{article}
\usepackage{../coursenotes}

\title{MAT 472 - Intermediate Real Analysis I}
\author{
{\Large Instructor: Dr. Steven Kaliszewski} \\
		Notes written by Brett Hansen
}
\date{}

\begin{document}

\maketitle
\tableofcontents
\break

\section{Week of August 21st, 2016}
\subsection{The Completeness Property of Real Numbers}
$\mathbb{R}$ and $\mathbb{Q}$ are ordered fields.
\begin{itemize}
\item $\ssor\mathbb{Q}\ssor < \ssor\mathbb{R}\ssor$
\item $\mathbb{Q}$ is countable, $\mathbb{R}$ is uncountable (cardinality)
\item $x^2-2=0$ has solutions in $\mathbb{R}$ but none in $\mathbb{Q}$
\item $\mathbb{R}$ has no gaps (order)
\end{itemize}

\subsubsection{Axiom of Completeness}
Every non-empty subset of $\mathbb{R}$ that has an upper bound has a least upper bound. Along with the greatest lower bound, these bounds can be classified using the following definitions: \\

\begin{defn}[Upper Bound]Let the set $S$ be a non-empty subset of $\mathbb{R}$ and suppose $b\in\mathbb{R}$. $b$ is an upper bound of the set $S$ if $\forall x\in S,\ x \leq b$.
\end{defn}
\begin{defn}[Lower Bound]Let the set $S$ be a non-empty subset of $\mathbb{R}$ and suppose $b\in\mathbb{R}$. $b$ is a lower bound of the set $S$ if $\forall x\in S,\ x \geq b$.
\end{defn}
\begin{defn}[Supremum]Let $B$ be the set of upper bounds of the non-empty subset, $S$, of $\mathbb{R}$ and suppose $b\in B$. If $\forall x\in B,\ b \leq x$, then $b$ is the least upper bound of $S$, or \textit{supremum} of $S$. This can be denoted as $\text{sup}\ S=b$.
\end{defn}
\begin{defn}[Infinum]Let $B$ be the set of lower bounds of the non-empty subset, $S$, of $\mathbb{R}$ and suppose $b\in B$. If $\forall x\in B,\ b \geq x$, then $b$ is the greatest lower bound of $S$, or \textit{infinum} of $S$. This can be denoted as $\text{inf}\ S=b$.
\end{defn}
\example{$A=[0,\ 2]\quad\longrightarrow\quad\text{sup}\ A=2$ \\
		   $A=[0,\ 2)\quad\longrightarrow\quad\text{sup}\ A=2$}

\begin{defn}[Maximum]If the supremum of a set is also a member of the set, then the supremum is also the maximum.
\end{defn}
% \example{Let $A=\{1/n\ssor n\in\mathbb{N}\}$, then $\text{max}\ A=\text{sup}\ A=1$ and $\text{inf}\ A=0$, however there is no minimum.}

% \noindent There are non-empty bounded subsets of $\mathbb{Q}$ with no supremum (or infinum) in $\mathbb{Q}$.
% \example{Let $S=\{r\in\mathbb{Q}\ssor r^2<2\}$. There does not exist any $b\in\mathbb{Q}$ such that $b \leq c\ \text{where}\ c\in\text{UB}\{S\}\ \text{and}\ c\in\mathbb{Q}$.}

% \proof{$S$ is non-empty and bounded above by $2$. Suppose $x\in\mathbb{Q}$ and $x>2$. Then $0\leq 2<x$ so $0\leq 2\cdot 2<2x$ and $0\leq 2x<x^2$. Then $2<4<x^2$ and thus $x\notin S$.}

\section{Week of August 28th, 2016}
\section{Week of September 4th, 2016}
\section{Week of September 11th, 2016}
\section{Week of September 18th, 2016}
\section{Week of September 25th, 2016}
\section{Week of October 2nd, 2016}
\section{Week of October 9th, 2016}
\section{Week of October 16th, 2016}
\section{Week of October 23rd, 2016}
\section{Week of October 30th, 2016}
\section{Week of November 6th, 2016}
\section{Week of November 13th, 2016}
\section{Week of November 20th, 2016}
\section{Week of November 27th, 2016}

\end{document}