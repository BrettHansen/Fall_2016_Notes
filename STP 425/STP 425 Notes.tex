\documentclass{article}
\usepackage{../coursenotes}

\title{STP 425 - Stochastic Processes}
\author{
{\Large Instructor: Dr. Nicolas Lanchier} \\
		Notes written by Brett Hansen
}
\date{}

\begin{document}

\maketitle
\tableofcontents
\break

\section{Week of August 14th, 2016}

\textbf{Stochastic Process} \quad a collection of random variables

\subsection{Basic Probability Review}
\subsubsection{Example 1.12}
Compute the probability that event E occurs before event F if we repeat the experiment.
\begin{align*}
p &&= &&\prob{E} &&+ &&\prob{(E \cup F)^C} \cdot p \\
  &&= &&\prob{E} &&+ &&1 - \prob{E \cup F} \cdot p \\
  &&= &&\prob{E} &&+ &&(1 - \prob{E} - \prob{F}) \cdot p \\
  &&= &&&&&&\frac{\prob{E}}{\prob{E} + \prob{F}}
\end{align*}

\subsubsection{Partitions}
If $\{B_1, B_2, \ellipsis, B_n\}$ is a partition of $\Omega$ and $A \subset \Omega$ then,
\begin{align*}
\prob{A} 	&&= &&&\prob{\bigcup_{i=1}^n A \cap B_i} \\
			&&= &&&\sum_{i=1}^n \left(\prob{A \cap B_i}\right) \\
			&&= &&&\sum_{i=1}^n \conprob{A}{B_i}\prob{B_i}
\end{align*}

\subsubsection{Example 1.13 - Craps}
Roll two dice, then if
\begin{enumerate}
	\item the sum is 7 or 11 $\longrightarrow$ you win!
	\item the sum is 2, 3, or 12 $\longrightarrow$ you lose!
	\item the sum, $i$, is such that $i \in \{4, 5, 6, 8, 9, 10\}$, keep rolling until
	\begin{enumerate}
		\item the sum is 7 $\longrightarrow$ you lose!
		\item the sum is $i$ $\longrightarrow$ you win!
	\end{enumerate}
\end{enumerate}

\noindent Let $W$ be the event that you win and $D$ the sum of the two dice.
\begin{align*}
\prob{W} 	&&= &&&&&&&&&&&\sum_{i=2}^{12} \left(\conprob{W}{D=i}\prob{D=i}\right) \\
			&&= &&&\prob{D=7} &&+ &&\prob{D=11} &&+ &&\sum_{i \in \{4,5,6,8,9,10\}} \left(\conprob{W}{D=i}\prob{D=i}\right) \\
			&&= &&&\frac{6}{36} &&+ &&\frac{2}{36} &&+ &&\sum_{i \in \{4,5,6,8,9,10\}} \left(\frac{\prob{D=i}^2}{\prob{D=i}+\prob{D=7}}\right) \\
			&&= &&&\frac{6}{36} &&+ &&\frac{2}{36} &&+ &&2\left(\frac{1}{36} + \frac{4}{90} + \frac{25}{396}\right) \\
			&&= &&&\frac{4880}{9900} &&\approx &&0.4929
\end{align*}

\section{Week of August 21st, 2016}
Let $X$ represent the number of times that a coin changes between heads and tails when flipped with the probability of landing on heads for any given flip $p$. Find the $pmf$ of $X$ when $p=1/2$. All outcomes can be expressed as having probability $(1/2)^n$.
$$X: \Omega\longrightarrow\{0,1,2,\ellipsis,n-1\}$$
$$\prob{X=k}=2\ \binom{n-1}{k}\ \frac{1}{2}=\binom{n-1}{k}\ \left(\frac{1}{2}\right)^{n-1}\ (?)$$
It is difficult to find $\expect{X}$ from the $pmf$, especially when $p\neq1/2$.

\begin{example}[2.4.1]
Find $\expect{X}$ for a general $p$. Start by writing $X$ as a sum of simple random variables.
$$X=\parsum{i=1}{n-1}{X_i}\text{ where } X_i=(?)\mathbb{L}\{\text{flip } i\neq\text{flip } i+1\}$$
$$\expect{X} = \expect{\parsum{i=1}{n-1}{X_i}} =\parsum{i=1}{n-1}{\expect{X_i}} = \parsum{i=1}{n-1}{\prob{X_i=1}} = \parsum{i=1}{n-1}{2p(1-p)} = 2p(p-1)(n-1)$$
\end{example}

\subsubsection{Coupon Collection Problem}
There are $m$ different types of coupons. Each time you purchase a coupon it is equally likely to be of any type. Let $X$ be the number of coupons we need to purchase to get a coupon of every type and let $X_i$ be the number of coupons needed to get $i$ different coupon types. Then $X=X_m$. Notice that $X_{i+1}-X_i$ is geometric with $\frac{m-i}{m}$ (?).
$$X=X_m=(X_m-X_{m-1})+(X_{m-1}-X_{m-2})+\midellipsis+(X_1-X_0)$$
\begin{align*}
\expect{X}&=\expect{\parsum{i=0}{m-1}{X_{i+1}-X_i}} \\
		  &=\parsum{i=0}{m-1}{\expect{X_{i+1}-X_i}} \\
		  &=\parsum{i=0}{m-1}{\frac{m}{m-i}} \\
		  &=\parsum{i=0}{m}{\frac{m}{i}} \\
		  &=m\left(\frac{1}{2}+\frac{1}{3}+\frac{1}{4}+\midellipsis+\frac{1}{n}\right)
\end{align*}
When $m$ is large, then this value is approximately $m\ \text{log}(m)$.

\subsubsection{Problem 1.32}
Suppose there are $n$ people and $n$ hats. Each person selects a hat at random. Find the probability of $A=$ nobody gets their hat. Let $A_i=$ person $i$ gets their hat. Then,
\begin{align*}
A&=A_1^C \cap A_2^C \cap \midellipsis \cap A_n^C \\
 &=\left(A_1 \cup A_2 \cup \midellipsis \cup A_n\right)^C
\end{align*}
Using the inclusion-exclusion identity we get:
\begin{align*}
&\prob{A}&&=1-\prob{\parunion{i=1}{n}{A_i}} \\
&		 &&=1-\parsum{i=1}{n}{(-1)^{i+1}\ \binom{n}{i}\prob{A_1\cap A_2\cap \midellipsis \cap A_n}} \\
&		 &&=1-\parsum{i=1}{n}{(-1)^{i+1}\ \binom{n}{i}\ \frac{(n-i)!}{n!}} \\
&		 &&=1-\parsum{i=1}{n}{(-1)^{i+1}\ \binom{n}{i}\ \frac{1}{i!}} \\
&		 &&=1+\parsum{i=1}{n}{(-1)^i\ \binom{n}{i}\ \frac{1}{i!}} \\
&		 &&=\parsum{i=0}{n}{(-1)^i\ \binom{n}{i}\ \frac{1}{i!}} \\
&\text{lim}_{n\rightarrow\infty}\left(\parsum{i=0}{n}{(-1)^i\ \binom{n}{i}\ \frac{1}{i!}}\right)&&=e^{-1}
\end{align*}

\subsubsection{Problem 2.72}
Same story as in Problem 1.32.
$$\prob{X=0}=\prob{A}=\parsum{i=0}{n}{(-1)^i\ \frac{1}{i!}}$$
Find $\expect{X}$ and $\var{X}$. We do not know the \textit{pmf}.
Define
$$X_i=
	\begin{cases} 
	      0 & \text{if person }i\text{ does not get their hat} \\
	      1 & \text{otherwise}
	\end{cases}$$ and let
$$X=X_1+X_2+\midellipsis+X_n\ .$$
$$\expect{X}=\expect{\parsum{i=1}{n}{X_i}}=\parsum{i=1}{n}{\expect{X_i}}=n\ \expect{X_1}=n\ \prob{X_1=1}=n\ \prob{A_1}=n\ \frac{1}{n}=1$$
\begin{align*}
\expect{X^2}=\expect{\left(\parsum{i=1}{n}{X_i}\right)^2}=\expect{\parsum{i=1}{n}{X_i^2}+\parsum{i\neq j}{}{X_i\ X_j}}&=&&\parsum{i=1}{n}{\expect{X_i^2}}&+&\parsum{i\neq j}{}{X_i\ X_j} \\
	&=&&n\ \frac{1}{n}&+&\parsum{i\neq j}{}{\prob{X_i=X_j=1}} \\
	&=&&1			  &+&\ (n^2-n)\prob{X_1=X_2=1} \\
	&=&&1			  &+&\ \frac{(n^2-n)}{n(n-1)} \\
	&=&&2
\end{align*}
$$\var{X}=\expect{\left(X-\expect{X}\right)^2}=\expect{X^2}-\expect{X}^2=2-1=1$$
\section{Week of August 28th, 2016}
\section{Week of September 4th, 2016}
\section{Week of September 11th, 2016}
\section{Week of September 18th, 2016}
\section{Week of September 25th, 2016}
\section{Week of October 2nd, 2016}
\section{Week of October 9th, 2016}
\section{Week of October 16th, 2016}
\section{Week of October 23rd, 2016}
\section{Week of October 30th, 2016}
\section{Week of November 6th, 2016}
\section{Week of November 13th, 2016}
\section{Week of November 20th, 2016}
\section{Week of November 27th, 2016}

\end{document}