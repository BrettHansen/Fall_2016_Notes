\documentclass{article}
\usepackage[margin=1in]{geometry}
\usepackage{multicol}
\usepackage{amsmath}
\usepackage{amsfonts}
\usepackage{hyperref}

\makeatletter
\newcommand{\ellipse}{\cdot\cdot\cdot}
\newcommand{\sor}{\quad | \quad}
\newcommand{\ssor}{\ |\ }
\newcommand{\sdot}{\quad\cdot\quad}
\newcommand{\prob}[1]{\mathbb{P}\left(#1\right)}
\newcommand{\conprob}[2]{\prob{#1\ssor#2}}

\title{CSE 415 Introduction to Combinatorics}
\author{Brett Hansen}
\date{}

\begin{document}

\maketitle
\tableofcontents
\break

\section{Week of August 14th, 2016}
\subsection{Principle Definitions}
\subsubsection{Product Principle}
Suppose a task can be broken into $k$ subtasks, $t_1,t_2,\ellipse,t_k$, and further suppose there are $c_i$ ways to perform subtask $t_i$ and each way leads to an unique result. Then the number of ways to perform the task is $c_1 \cdot c_2 \ellipse c_k$.

\subsubsection{Sum Principle}
Suppose the objects in a counting problem can be divided into $k$ disjoint and exhaustive cases. If there are $n_i$ objects in the $i^{th}$ case for $i=1,2,\ellipse,k$ then there are $n_1+n_2+\ellipse+n_k$ objects.

\subsubsection{Bijection Principle}
Two finite sets have the same cardinality if and only if there exists a bijection between them. \\
\newline
\textbf{Example} \quad How many subsets does $\{k_1,k_2,k_3,k_4\}$ have? \\
Find a bijection between the binary string $b_1b_2b_3b_4$ and $\{k_1,k_2,k_3,k_4\}$. \\

$$S\subseteq\{k_1,k_2,k_3,k_4\}\longleftrightarrow b_1b_2b_3b_4 \quad\text{where}\quad b_i=
\begin{cases} 
	0 & \text{if}\quad k_i \notin S\\
	1 & \text{if}\quad k_i \in S
\end{cases}
$$

\noindent There are $2^4=16$ possibilites for the binary string so the set has $16$ subsets.

\subsubsection{Quotient Principle}
A \textit{partition} of a set, $S$, is a division of a set into disjoint subsets whose union is $S$. The subsets in a set of partitions are often called blocks of the partition. \newline
\noindent Suppose a set $S$ has $p$ elements. If we partition $S$ into $q$ blocks of size $r$, then $q=p/r$ and $r=p/q$.

\end{document}