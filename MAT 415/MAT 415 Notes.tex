\documentclass{article}
\usepackage{../coursenotes}

\title{MAT 415 - Introduction to Combinatorics}
\author{
{\Large Instructor: Dr. Susanna Fishel} \\
		Notes written by Brett Hansen
}
\date{}

\begin{document}

\maketitle
\tableofcontents
\break

\section{Week of August 14th, 2016}
\subsection{Principle Definitions}
\subsubsection{Product Principle}
Suppose a task can be broken into $k$ subtasks, $t_1,t_2,\ellipsis,t_k$, and further suppose there are $c_i$ ways to perform subtask $t_i$ and each way leads to an unique result. Then the number of ways to perform the task is $c_1 \cdot c_2 \midellipsis c_k$.

\subsubsection{Sum Principle}
Suppose the objects in a counting problem can be divided into $k$ disjoint and exhaustive cases. If there are $n_i$ objects in the $i^{th}$ case for $i=1,2,\ellipsis,k$ then there are $n_1+n_2+\ellipsis+n_k$ objects.

\subsubsection{Bijection Principle}
Two finite sets have the same cardinality if and only if there exists a bijection between them. \\
\newline
\textbf{Example} \quad How many subsets does $\{k_1,k_2,k_3,k_4\}$ have? \\
Find a bijection between the binary string $b_1b_2b_3b_4$ and $\{k_1,k_2,k_3,k_4\}$. \\

$$S\subseteq\{k_1,k_2,k_3,k_4\}\longleftrightarrow b_1b_2b_3b_4 \quad\text{where}\quad b_i=
\begin{cases} 
	0 & \text{if}\quad k_i \notin S\\
	1 & \text{if}\quad k_i \in S
\end{cases}
$$

\noindent There are $2^4=16$ possibilites for the binary string so the set has $16$ subsets.

\subsubsection{Quotient Principle}
A \textit{partition} of a set, $S$, is a division of a set into disjoint subsets whose union is $S$. The subsets in a set of partitions are often called blocks of the partition. \newline
\noindent Suppose a set $S$ has $p$ elements. If we partition $S$ into $q$ blocks of size $r$, then $q=p/r$ and $r=p/q$.

\section{Week of August 21st, 2016}
\subsection{Binomial Coefficients and the Quotient Principle}
Let $P$ of size $p$ be the set of all permutations of $S=\{1,2,\ellipsis,n\}$. We say that two permutations are in the same block if they have the same first $k$ values. Then $p=n!$ and $q=k!$ (?).
$$\binom{n}{k}=\frac{n!}{k!(n-k)!} \text{ is the number of }k\text{-element subsets of an }n\text{-element set.}$$

\subsection{Lattice Paths}
A lattice path is a path from $(a,b)$ to $(m,n)$ that only takes steps of length 1 in north or east directions. Thus, the number of north steps is $n-b$ and the number of east steps is $m-a$. So the total steps required is $n-b+m-a$. \\

\noindent Since a step is either east or north, we can uniquely identify all lattice paths by only considering when an east step is taken. For example, between $(0,0)$ and $(2,1)$, we must take $2$ steps to the east. Thus out of our $3$ total steps the set of all unique sets of when we can take steps to the east is $\{\{1,2\},\{1,3\},\{2,3\}\}$. The missing member of each subset is when we take north steps and thus we have identified all unique lattice paths betweeen $(0,0)$ and $(2,1)$. In general, the number of unique lattice paths between $(a,b)$ and $(m,n)$ is:
$$\binom{n-b+m-a}{m-a}=\binom{n-b+m-a}{n-b}$$

\subsubsection{Catalan Numbers}
If we only consider paths between $(0,0)$ and $(n,n)$, then we can define \textit{good paths} as those that do not touch the line $y=x+1$ which is equivalent to those paths that do not cross $y=x$. \textit{Bad paths} are simply paths that are not good. A bijection exists between bad paths, and all paths between $(-1,1)$ and $(n,n)$ (see notes). Thus the number of bad paths is simply:
$$\binom{n-1+n+1}{n+1}=\binom{2n}{n+1}$$

\noindent The number of good paths is then the subtraction of the total number of paths and the number of bad paths:
\begin{align*}
\binom{n-0+n-0}{n-0}-\binom{2n}{n+1}=\binom{2n}{n}-\binom{2n}{n+1}&=\frac{(2n)!}{n!\ n!}-\frac{(2n)!}{(n+1)!\ (n-1)!} \\
		&=\frac{(2n)!\ (n+1)-(2n)!\ n}{(n+1)!\ n!} \\
		&=\frac{(2n)!}{(n+1)!\ n!} \\
\end{align*}

\noindent The values this expression generates are called the Catalan numbers.

\subsubsection{The Pigeonhole Principle}
Suppose that there are $7$ houses and $10$ people to place in these houses. Obviously not everyone can have their own house.
\begin{corl}
A function from a set with $k+1$ elements to a set with $k$ elements cannot be one-to-one.
\end{corl}
\begin{example}[California Example]
Suppose we label boxes with three letters representing a person's initials and a day of the year representing their birthday. This generates $26\cdot 26\cdot 26\cdot 365=6415240$ boxes. Given that there are about $38.8$ million people in California, if we assign each person a box based on their initials and birthday, there must be a box shared by at least $\ceil{\frac{N}{k}}=\ceil{6.048}=7$ people.
\end{example}
\section{Week of August 28th, 2016}
\section{Week of September 4th, 2016}
\section{Week of September 11th, 2016}
\section{Week of September 18th, 2016}
\section{Week of September 25th, 2016}
\section{Week of October 2nd, 2016}
\section{Week of October 9th, 2016}
\section{Week of October 16th, 2016}
\section{Week of October 23rd, 2016}
\section{Week of October 30th, 2016}
\section{Week of November 6th, 2016}
\section{Week of November 13th, 2016}
\section{Week of November 20th, 2016}
\section{Week of November 27th, 2016}

\end{document}